\chapter{Аналитическая часть}

В этом разделе будет рассмотрен криптографический алгоритм RSA.

\section{Алгоритм RSA}

RSA (аббревиатура от фамилий Rivest, Shamir и Adleman) --- ассиметричный алгоритм с открытым ключом, основывающийся на вычислительной сложности задачи факторизации больших полупростых чисел. В алгоритме RSA используется 2 ключа --- открытый (публичный) и закрытый (приватный).

В ассиметричной криптографии и алгоритме RSA, в частности, открытый и закрытый ключи являются двумя частями одного целого и неразрывны друг с другом. Для шифрования информации используется открытый ключ, а для её расшифровки закрытый.


RSA ключи генерируются следующим образом:
\begin{enumerate}[label=\arabic*)]
	\item выбираются два отличающихся друг от друга случайных простых числа $p$ и $q$, лежащие в установленнном диапазоне;
	\item вычисляется их произведение $n = p \cdot q$, называемое модулем;
	\item вычисляется значение функции Эйлера от числа $n$: $\phi(n) = (p - 1)\cdot (q - 1)$;
	\item выбирается целое число $e$ ($1 < e < \phi(n)$), взаимно простое со значением $\phi(n)$, оно называется открытой экспонентой;
	\item вычисляется число $d \cdot e  \equiv 1 (mod (\phi(n))$, оно называется закрытой экспонентой.
\end{enumerate}

Пара $(e, n)$ публикуются в качестве открытого ключа RSA, а пара $(d, n)$ --- в виде закрытого ключа.


Шифрование сообщения $m$ ($0 < m < n - 1)$ в зашифрованное сообщение $c$ производится по формуле $ c = E(m, k_1) = E(m, n, e) = m^{e} mod (n)$.

Расшифровка: $m = D(c, k_2) = D(c, n, d) = c^{d} mod (n)$
