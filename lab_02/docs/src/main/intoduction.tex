\chapter*{ВВЕДЕНИЕ}
\addcontentsline{toc}{chapter}{ВВЕДЕНИЕ}

Шифрование с открытым ключом, в отличие от симметричного шифрования, использует два различных ключа — открытый и закрытый, что решает проблему безопасной передачи ключа. Алгоритм RSA --- один из первых и наиболее широко применяемых алгоритмов асимметричного шифрования.

\textbf{Цель лабораторной работы:} проектирование и разработка программной реализации алгоритма шифрования RSA. Для достижения поставленной цели необходимо выполнить следующие задачи:

\begin{itemize}[label=---]
	\item провести анализ работы алгоритма RSA;
	\item описать алгоритм шифрования с открытым ключом;
    \item реализовать и протестировать реализацию алгоритма шифрования.
\end{itemize}
